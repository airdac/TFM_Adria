\documentclass[a4paper,12pt]{article}
\usepackage[left=2.5cm, right=2.5cm, top=2.5cm, bottom=2.5cm]{geometry}
\usepackage{pdfpages}
\usepackage{amssymb}
\usepackage{amsmath}
\usepackage{float}
\usepackage{csquotes}
\usepackage{subcaption}
\usepackage[backend=biber]{biblatex}
\addbibresource{../bibliography.bib}
\usepackage{minted}
\setminted{frame=lines, fontsize=\small}
\usepackage{graphicx}
\usepackage{hyperref}
\hypersetup{
    colorlinks=true,
    linkcolor=black,
    filecolor=black,
    urlcolor=black,
    citecolor=black
}
\usepackage{algorithm}
\usepackage{algorithmic}
% Allow more flexible breaking of long URLs:
\setlength\emergencystretch{3em}
\setcounter{biburllcpenalty}{7000}
\setcounter{biburlucpenalty}{7000}
\Urlmuskip=0mu plus 1mu

\title{Distance-based dimensionality reduction for big data}
\author{Adrià Casanova Lloveras}

\begin{document}

\includepdf{front_cover.pdf}

\begin{abstract}
    Dimensionality reduction aims to project a data set into a low-dimensional space. Many techniques have been proposed, most of them based on the inter-individual distance matrix. When the number of individuals is really large, the use of distance matrices is prohibitive. There are algorithms that extend MDS (a classical dimensionality reduction method based on distances) to the big data setting. In this TFM, we adapt these algorithms to any generic distance-based dimensionality reduction method.
\end{abstract}
\pagebreak

\tableofcontents
\pagebreak

\section{Introduction, Motivation, and Objectives}

\subsection{Introduction}

\begin{itemize}
    \item Dimensionality Reduction definition, goal and applications.
    \item Examples of DR methods.
    \item Key points and limitations of DR methods.
\end{itemize}

\subsection{Motivation}

\begin{itemize}
    \item When the number of individuals is really large, the use of distance matrices is prohibitive.
    \item There are algorithms that extend MDS to the big data setting.
\end{itemize}

\subsection{Objectives}

\begin{itemize}
    \item Adapt these algorithms to any generic distance-based dimensionality reduction method.
\end{itemize}
\pagebreak

\section{State of the Art}

\subsection{Introduction}

\begin{itemize}
    \item There are many DR algorithms, linear and non-linear, but they use the distance matrix of datapoints. In big datasets, this matrix cannot fit in the system's RAM, so DR methods are not feasible. Moreover, time complexity can be prohibitive in some cases as well.
    \item Delicado and Pachón-García proposed new versions of MDS that handled this problem and compared them with prior algorithms \cite{Delicado2024MDSBigData}.
    \item Regarding non-linear methods, Landmark Isomap \cite{deSilvaTenenbaum2002} was proposed to adapt Isomap to large data settings. Later, in 2024, Reichmann, Hägele and Weiskopf generalized Interpolation MDS to any DR method that would return a map between high- and low-dimensional spaces, such as PCA, non-classicla MDS, t-SNE, UMAP or Autoencoders.
    \item Finally, t-SNE has been optimized for big data environments in Python through iterative implementations. Their details, however, are out of the scope of this project.
\end{itemize}

\subsection{A few Dimensionality Reduction Techniques}

\subsubsection{Non-classical MDS. The SMACOF algorithm}

SMACOF (Scaling by MAjorizing a COmplicated Function) is a multidimensional scaling algorithm that minimizes metric stress using a majorization technique \cite{borg1997modern}. Also known as the Guttman Transform, this technique is more powerful for this problem than general optimization methods, such as gradient descent.

\begin{algorithm}
    \caption{SMACOF}
    \label{alg:SMACOF}
    
    \begin{algorithmic}[1]
    \REQUIRE $D_{\mathcal{X}} = (\delta_{ij})$, the matrix of observed distances; $q$, the embedding's dimensionality; $n\_iter$, the maximum number of iterations; and $\epsilon$, the convergence threshold.
    \ENSURE $\tilde{\mathcal{Y}}$, a configuration in a $q$-dimensional space.
    \STATE Initialize $\tilde{\mathcal{Y}}^{(0)} \in \mathbb{R}^{n \times q}$
    \STATE $k \leftarrow 0$
    \REPEAT
        \STATE Compute distance matrix of $\tilde{\mathcal{Y}}^{(k)}$:  $d_{ij}(\tilde{\mathcal{Y}}^{(k)}) = \|x_i - x_j\|$
        \STATE Compute the Metric STRESS: $STRESS_M(D_{\mathcal{X}}, \tilde{\mathcal{Y}}^{(k)}) = \sqrt{\frac{\sum_{i<j}\left(\delta_{i j}-d_{i j}\right)^2}{\sum_{i<j} \delta_{i j}^2}}$
        \STATE Compute the Guttman Transform: $\tilde{\mathcal{Y}}^{(k+1)} = n^{-1}B(\tilde{\mathcal{Y}}^{(k)})\tilde{\mathcal{Y}}^{(k)}$ where $B(\tilde{\mathcal{Y}}^{(k)}) = (b_{ij})$:
        $$
        b_{ij} =
        \begin{cases}
        -\delta_{ij}/d_{ij}(\tilde{\mathcal{Y}}^{(k)}) & \text{if } i \neq j \text{ and } d_{ij}(\tilde{\mathcal{Y}}^{(k)}) > 0 \\
        0 & \text{if } i \neq j \text{ and } d_{ij}(\tilde{\mathcal{Y}}^{(k)}) = 0 \\
        -\sum_{j \neq i} b_{ij} & \text{if } i = j
        \end{cases}
        $$
        \STATE $k \leftarrow k + 1$
    \UNTIL{$k \geq n\_iter$ or $|STRESS_M(D_{\mathcal{X}}, \tilde{\mathcal{Y}}^{(k-1)}) - STRESS_M(D_{\mathcal{X}}, \tilde{\mathcal{Y}}^{(k)}| < \epsilon$}
    \RETURN $\tilde{\mathcal{Y}}^{(k)}$
    \end{algorithmic}
\end{algorithm}

\subsubsection{Local MDS}

Local MDS \cite{LocalMDS} is a variant of non-classical multidimensional scaling that differs in how large distances are treated. Specifically, a repulsive term between distant points is added to the stress function to further separate points in the low-dimensional configuration.

\begin{algorithm}
    \caption{Local MDS}
    \label{alg:LocalMDS}
    
    \begin{algorithmic}[1]
    \REQUIRE $D_{\mathcal{X}}$, the matrix of observed distances; $q$, the embedding's dimensionality; $k$, the size of neighborhoods; and $\tau$, the weight of the repulsive term.
    \ENSURE $\tilde{\mathcal{Y}}$, a configuration in a $q$-dimensional space.
    \STATE Compute the symmetrized k-NN graph of $D_{\mathcal{X}}$, $\mathcal{N}$
    \STATE Calculate $t=\frac{|\mathcal{N}|}{\left|\mathcal{N}^C\right|} \cdot \operatorname{median}_{\mathcal{N}}\left(D_{i, j}\right) \cdot \tau$
    \STATE Minimize $$\sum_{(i, j) \in N}\left(D_{i, j}-\left\|\mathbf{y}_i-\mathbf{y}_j\right\|\right)^2 - t \sum_{(i, j) \notin N}\left\|\mathbf{y}_i-\mathbf{y}_j\right\|$$
    \RETURN The solution to the optimization problem, $\tilde{\mathcal{Y}}$.
    \end{algorithmic}
\end{algorithm}

Parameters $\tau$ - which must be in the unit interval - and $k$ may be tuned with $k'$-cross validation thanks to the LCMC (Local Continuity Meta-Criteria). (\textit{POSSIBLE ANNEX})

\subsubsection{Isomap}

Isomap \cite{Tenenbaum2000} is a nonlinear technique that preserves geodesic distances between points in a manifold. The key insight of Isomap is that large distances between objects are estimated from the shorter ones by the shortest path length. Then, shorter and estimated-larger distances have the same importance in a final MDS step.

\begin{algorithm}
    \caption{Isomap}
    \label{alg:Isomap}

    \begin{algorithmic}[1]
    \REQUIRE $D_{\mathcal{X}}$, the matrix of observed distances; $q$, the embedding's dimensionality; and $\epsilon$ or $k$, the bandwith.
    \ENSURE $\tilde{\mathcal{Y}}$, a configuration in a $q$-dimensional space.
    \STATE Find the $\epsilon$-NN or k-NN graph of $\mathcal{X}$, $G$.
    \STATE Compute the distance matrix of $G$, $D_G$.
    \STATE Embed $D_G$ to a $q$-dimensional space with MDS.
    \RETURN The output configuration of MDS.
    
    \end{algorithmic}
\end{algorithm}

The only tuning parameter of Isomap is the bandwith ($\epsilon$ or $k$), but there is no consensus on what is the best method to choose it.

\subsubsection{t-SNE}

t-SNE (t-Distributed Stochastic Neighbor Embedding) \cite{tsne} is a nonlinear dimensionality reduction technique that preserves local neighborhoods by modeling similarities between points as conditional probabilities. The difference between these probability distributions in high and low-dimensional spaces is then minimized.

\begin{algorithm}
    \caption{t-SNE}
    \label{alg:tSNE}
    
    \begin{algorithmic}[1]
    \REQUIRE $\mathcal{X} \in \mathbb{R}^{n \times p}$, the high-dimensional configuration; $q$, the embedding's dimensionality; perplexity $Perp$.
    \ENSURE $\tilde{\mathcal{Y}}$, a configuration in a $q$-dimensional space.
    \STATE For every datapoint $i$, find $\sigma_i$ so that the conditional probability ditribution
        $$p_{j|i} = \frac{\exp(-\|x_i-x_j\|^2/2\sigma_i^2)}{\sum_{k \neq i}\exp(-\|x_i-x_k\|^2/2\sigma_i^2)}$$ has perplexity $2^{-\sum_{j}p_j \log_2 p_j} = Perp$
    \STATE Symmetrize conditional distributions: $p_{ij} = \frac{p_{j|i} + p_{i|j}}{2n}$ if $i\neq j$, $p_{ii} = 0$
    \STATE Consider Student t-distributed joint probabilities for the low-dimensional data $y_i$: $$q_{ij} = \frac{(1 + \|y_i-y_j\|^2)^{-1}}{\sum_{h \neq k}(1 + \|y_h-y_k\|^2)^{-1}}$$
    \STATE Minimize the sum of Kullback-Leibler divergences between the joint distributions over all datapoints: $$C(\tilde{\mathcal{Y}})=\sum_i \sum_j p_{j \mid i} \log \frac{p_{j \mid i}}{q_{j \mid i}}$$
    \RETURN The solution to the optimization problem, $\tilde{\mathcal{Y}}$.
    
    \end{algorithmic}
\end{algorithm}

t-SNE focuses on retaining the local structure of the data while ensuring that every point $y_i$ in the low-dimensional space will have the same number of neighbors, making it particularly effective for visualizing clusters. The use of the Student t-distribution in the low-dimensional space addresses the \textit{crowding problem} by allowing dissimilar points to be modeled far apart \cite{tsne}.

Perplexity is interpreted as the average effective number of neighbors of the high-dimensional datapoints $x_i$ and typical values are between 5 and 50.

\subsection{Multidimensional Scaling for Big Data}

Delicado and Pachón-García \cite{Delicado2024MDSBigData} compared four existing versions of MDS with two newly proposed (Divide-and-conquer MDS and Interpolation MDS) to handle large data. As can be seen in figure \ref{fig:bigmds}, these can be grouped into four categories:

\begin{itemize}
    \item \textbf{Interpolation-based}: Landmark MDS, Interpolation MDS and Reduced MDS apply classical multidimensional scaling to a subset of $l \ll n$ points and then interpolate the projection of the remaining data. They differ in how the interpolation is computed: Landmark MDS uses distance-based triangulation; Interpolation MDS, the $l$-points Gower Interpolation Formula; and Reduced MDS, the 1-point Gower Interpolation Formula.
    \item \textbf{Approximation-based}: Pivot MDS approximates the SVD of the full inner product matrix with the SVD of the inner product matrix between a subset of $l \ll n$ points and all the points in the dataset.
    \item \textbf{Divide-and-conquer}: In Divide-and-conquer MDS, the dataset is randomly partitioned into subsets of up to $l \ll n$ points into which MDS is independently applied. Then, the resulting embeddings are aligned with Procrustes transformations.
    \item \textbf{Recursive}: Fast MDS is similar in spirit to Divide-and-conquer MDS, but it partitions the data recursively. 
\end{itemize}

\begin{figure}[ht]
    \centering
    \includegraphics[width=0.8\textwidth]{figures/bigmds.png}
    \caption{Schematic representation of the six MDS algorithms described by Delicado and Pachón-García \cite{Delicado2024MDSBigData}.}
    \label{fig:bigmds}
\end{figure}

\subsection{Landmark Isomap and the Out-of-Core Dimensionality Reduction Framework}

Silva and Tenenbaum \cite{deSilvaTenenbaum2002} first introduced Landmark MDS in 2002 by applying it to Isomap (L-Isomap). This way, they reduced the time complexity of both classical multidimensional scaling and Isomap from $\mathcal{O}(n^3)$ to $\mathcal{O}(n^2l)$, where $l \ll n$ is the amount of landmark points.

Note that, similarly to Landmark MDS, other big data versions of MDS can be used with Isomap. Nonetheless, interpolation-based and approximation-based algorithms cannot be trivially generalized to nonlinear dimensionality reduction methods.

Later, in 2024, Reichmann, Hägele and Weiskopf \cite{reichmann2024outofcoredimensionalityreductionlarge} proposed the Out-of-Core Dimensionality Reduction Framework. Similar to Interpolation MDS, this algorithm applies a DR method that produces a mapping between high- and low-dimensional spaces (i.e. PCA, MDS, t-SNE, UMAP, Autoencoder) to a small subset of the data and then projects the remaining datapoints in blocks. In order to obtain the aforementioned mappings, Reichmann et. al. gathered different projection mechanisms for every method. PCA and autoencoders learn a parametric mapping between the original data and the embedding, so projecting new points is straightforward. For non-classical MDS, stress is minimized for a single point while keeping others fixed, a process known as \textit{single scaling} in the literature \cite{single-scaling}. A similar strategy is used for t-SNE \cite{tsne-knn} and UMAP \cite{McInnes2018}, which leverage the k-NN of the projecting point to initizalize the optimizer.
\pagebreak

\section{Specification and Design of the Solution}

\begin{algorithm}
    \caption{Divide-and-Conquer for Dimensionality Reduction}
    \label{alg:DivideConquer}
    
    \begin{algorithmic}[1]
    \REQUIRE $\mathcal{X} \in \mathbb{R}^{n \times p}$, the high-dimensional data; $\mathcal{M}$, the DR method; $l$, the partition size; $c$, the amount of connecting points; $q$, the embedding's dimensionality; and $arg$, the $\mathcal{M}$'s specific parameters.
    \ENSURE $\tilde{\mathcal{Y}}$, a configuration in a $q$-dimensional space.
    
    \IF{$n \leq l$}
        \RETURN $\mathcal{M}(\mathcal{X}, q)$
    \ENDIF
    
    \STATE Partition data into $k$ subsets: $\mathcal{P} = \{\mathcal{P}_1, \mathcal{P}_2, \ldots, \mathcal{P}_k\}$ where $|\mathcal{P}_i| \leq l$ for all $i$
    
    \STATE Apply DR method to first partition: $\tilde{\mathcal{Y}}_1 = \mathcal{M}(\mathcal{X}_{\mathcal{P}_1}, q)$
    
    \STATE Randomly sample $c$ connecting points from $\mathcal{P}_1$: $\mathcal{C} \subset \mathcal{P}_1$ with $|\mathcal{C}| = c$

    \STATE Extract embedding of $\mathcal{C}$: $\tilde{\mathcal{Y}}_\mathcal{C} = \tilde{\mathcal{Y}}_1[{\mathcal{C}},:]$
    
    \FOR{$i = 2$ to $k$}
        \STATE Combine connecting points with current partition: $\mathcal{X}_{\text{join}} = [\mathcal{X}_{\mathcal{C}}; \mathcal{X}_{\mathcal{P}_i}]$
        \STATE Project combined data: $\tilde{\mathcal{Y}}_{\text{join}} = \mathcal{M}(\mathcal{X}_{\text{join}}, q)$
        \STATE Separate embedding of $\mathcal{X}_{\mathcal{C}}$: $\tilde{\mathcal{Y}}_{\mathcal{C}}^{(i)} = \tilde{\mathcal{Y}}_{\text{join}}[1:c,:]$ and $\tilde{\mathcal{Y}}_i = \tilde{\mathcal{Y}}_{\text{join}}[(c+1):,:]$
        \STATE Align projection using Procrustes: $\tilde{\mathcal{Y}}_i = \text{Procrustes}(\tilde{\mathcal{Y}}_{\mathcal{C}}^{(i)}, \tilde{\mathcal{Y}}_\mathcal{C}, \tilde{\mathcal{Y}}_i)$
    \ENDFOR
    
    \STATE Combine all projections: $\tilde{\mathcal{Y}}_{\text{combined}} = [\tilde{\mathcal{Y}}_1; \tilde{\mathcal{Y}}_2; \ldots; \tilde{\mathcal{Y}}_k]$
    \STATE Reorder rows to match original ordering: $\tilde{\mathcal{Y}}' = \tilde{\mathcal{Y}}_{\text{combined}}[\text{order},:]$
    \STATE Apply PCA to center and rotate for maximum variance: $\tilde{\mathcal{Y}}$ = PCA($\tilde{\mathcal{Y}}', q$)
    
    \RETURN $\tilde{\mathcal{Y}}$
    \end{algorithmic}
\end{algorithm}
\pagebreak

\section{Development of the Proposal}

\begin{itemize}
    \item DR methods implementations: packages used, problems found during development, tuning of parameters, experiments' methodology
\end{itemize}

\subsection{Python implementation of Divide-and-conquer for DR}

Given that R and Python are the standard programming languages in the Data Science field, we chose them as appropriate to implement the divide-and-conquer for DR algorithm. Initially, we aimed to depelop and publish an R library because the thesis directors already had experience with it. However, after reviewing the literature on DR for Big Data \cite{Reichmann2024}, we realized that many solutions were implemented in Python instead. So, in order to leverage the existing coding ecosystem, we switched to Python.

Our code, then, as well as the whole thesis, was documented on an open-source GitHub repository (\href{https://github.com/airdac/TFM_Adria}{https://github.com/airdac/TFM\_Adria}). This system will also allow us to update and share our implementations and experiments easily.

With time, python modules have been structured in a directory tree as a library. Therefore, even though our project has not been published in any Python package index, it effectively works as a library with specific classes and methods. The main function is \verb|divide_conquer|, which implements Algorithm \ref{alg:DivideConquer} in parallel through the \verb|concurrent.futures| module. \verb|divide_conquer| also depends on private methods and requires a \verb|DRMethod| object as one of its arguments. This class, inherited from \verb|enum.Enum|, lists the supported DR methods in our package, which can be called through the \verb|get_method_function| method.

We have implemented four DR techniques: SMACOF, Local MDS, Isomap and t-SNE. All but Local MDS are wrappers to methods in other Python libraries, which are efficient and parallelized. Specifically, we make us of the \verb|sklearn.manifold| module \cite{Pedregosa2011} for Isomap and SMACOF and \verb|openTSNE| \cite{Poličar2023} for t-SNE. Local MDS, on the other hand, is a less popular method and has no public Python implementation at the moment. However, the R library \verb|smacofx| (\textit{CITAR}) does. Therefore, we translated it to Python and adapted it to our framework. This also allowed us to optimize it with the \verb|numba| jit compitler (\textit{CITAR}).

\subsection{Python Packages Used}

\begin{itemize}
    \item \textbf{In general and for experiments}: \begin{itemize}
        \item numpy
        \item pandas
        \item matplotlib.pyplot
        \item os
        \item time
        \item sys
        \item logging
        \item warnings
        \item typing
        \item shutil
    \end{itemize}
    \item \textbf{In D\&C}: \begin{itemize}
        \item concurrent.futures for parallelization
    \end{itemize}

    \item \textbf{For DR methods}: \begin{itemize}
        \item enum
        \item scipy.spatial.distance
        \item sklearn.manifold.Isomap, sklearn.manifold.smacof
        \item openTSNE.TSNE
        \item numba, sklearn.neighbors and stops (from R translated to Python) for Local MDS
    \end{itemize}
\end{itemize}

\subsection{Experiment's methodology and tuning of parameters}

Experiments' methodology.

Tuning of parameters. A few cases.
\pagebreak

\section{Experimentation and evaluation of the proposal}

\begin{itemize}
    \item Swiss Roll.
    \item MNIST.
    \item Time complexity measurements (including MDS).
    \item Evaluation and comparison with state of the art.
\end{itemize}
\pagebreak

\section{Analysis of sustainability and ethical implications}

\textbf{DESCRIPTION OF THIS SECTION FROM THE REGULATION:}

It must include an analysis of the impact of the following gender-related technical aspects:

\begin{itemize}
    \item issues related to data management and analysis
    \item issues related to equity, where possible biases are identified and assessed both in the data and in the processes carried out in relation to data management and analysis
    \item actions carried out to eliminate or mitigate such biases
\end{itemize}

\textbf{ACTUAL CONTENT:}

\subsection{GHG emissions}

AI and ML polute a lot because of the raising data collection, transfer and computation in developed countries. There are three ways of reducing the impact: innovating algorithms, reducing data transfer and storing costs, making hardware more efficient. Our algorithm follows this trend and reduces from quadratic to linear the emissions of DR. CO2e is linear with respect to runtime, so we reduce the computation emission quadratically. Moreover, we make DR less dependant on data centers (which polute in its construction, maintanance on operation), since the reduction in space complexity allows standard computers to apply any distance-based DR technique.

Even though it is difficult to estimate the environmental cost of algortihms, we used the Green Algorithms calculator to provide some numbers based on compute hours and our Windows system. Table ??? shows them.

Even though there is still a cost to consider, given it will not be as used as chatbots and other neural networks, we are confident that our algorithm is sustainable and does not suppose a threat to the environment. Hence, using our algorithm would be benefficial to reduce carbon emissions as opposed to traditional DR methods.

\subsection{Invisibilization of small communities}

Maybe (it needs testing) DR methods could emphasize biases in the data. This happens because when projecting only a few coordinates, small clusters could be left behind in the remaining not projected coordinates and do not show in the final embedding. Hence, in theory, small communities could become invisible.
\pagebreak

\section{Conclusions}

The primary objective of this thesis was to develop and evaluate a generalized divide-and-conquer framework for distance-based dimensionality reduction methods to make them applicable to large datasets. We successfully met the objectives of this research by first reviewing the literature on prominent DR techniques and then developing a generalized framework that uses a divide-and-conquer strategy with orthogonal Procrustes transformations to reduce time and memory complexities. We implemented the framework in Python for non-classical MDS (SMACOF), LMDS, Isomap, and t-SNE, although it is easily extendable to any distance-based DR technique.

Later, we experimented with the framework by measuring its runtime and size limitations and assessing the embedding quality it provided on benchmark datasets. Results were mostly favorable, the most impressive one being the flawless embedding of a 100 million points Swiss roll on a standard computer in about 3 hours. In comparison, most bare DR methods would crash when trying to process datasets with more than 10,000 observations because of lacking memory. We also contrasted the embeddings of the bare and divide-and-conquer versions of SMACOF, LMDS, Isomap and t-SNE on smaller datasets. Results showed that divide-and-conquer DR was remarkably faster than bare methods while presenting small differences in the resulting embeddings. The only exception of this behaviour was found on t-SNE, since its \verb|openTSNE| implementation is very optimized and clearly outruns our framework both in runtime and embedding quality.

Ultimately, this work contributes to making advanced DR techniques more accessible and sustainable for the large-scale data challenges in science and industry by mitigating the prohibitive quadratic time and memory complexities of distance-based dimensionality reduction methods.

\subsection{Future work}

Based on the research and findings of this thesis, we have identified several avenues for future work. To start with, the implemented code could be formalized into a distributable Python package to make it more accessible. Moreover, a comparison could be conducted between our divide-and-conquer framework and the out-of-core approach by \citet{Reichmann2024}, since both reduce the space and time complexities of already established DR methods. In terms of less abstract tasks, we realized during the testing of LMDS that it performed worse than expected on the Swiss roll dataset. Therefore, further investigation on LMDS could lead to a better version of this algorithm that would handle the nonlinearities present in the Swiss roll dataset.

Regarding experimentation, future tests could explore how the number of connecting points, $c$, affects performance and embedding quality. Additionally, testing divide-and-conquer DR on more datasets and use cases would further validate its robustness and generalizability.
\pagebreak

\printbibliography
\pagebreak

\input{sections/8-annexes.tex}
\pagebreak


\end{document}