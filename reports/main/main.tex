\documentclass[a4paper,12pt]{article}
\usepackage[left=2.5cm, right=2.5cm, top=2.5cm, bottom=2.5cm]{geometry}
\usepackage{pdfpages}
\usepackage{amssymb}
\usepackage[backend=biber]{biblatex}
\addbibresource{../bibliography.bib}
\usepackage{minted}
\setminted{frame=lines, fontsize=\small}
\usepackage{graphicx}
\usepackage{hyperref}
\hypersetup{
    colorlinks=true,
    linkcolor=black,
    filecolor=black,
    urlcolor=black,
    citecolor=black
}
% Allow more flexible breaking of long URLs:
\setlength\emergencystretch{3em}
\setcounter{biburllcpenalty}{7000}
\setcounter{biburlucpenalty}{7000}
\Urlmuskip=0mu plus 1mu

\title{Distance-based dimensionality reduction for big data}
\author{Adrià Casanova Lloveras}

\begin{document}

\includepdf{front_cover.pdf}

\begin{abstract}
    Dimensionality reduction aims to project a data set into a low-dimensional space. Many techniques have been proposed, most of them based on the inter-individual distance matrix. When the number of individuals is really large, the use of distance matrices is prohibitive. There are algorithms that extend MDS (a classical dimensionality reduction method based on distances) to the big data setting. In this TFM, we adapt these algorithms to any generic distance-based dimensionality reduction method.
\end{abstract}
\pagebreak

\tableofcontents
\pagebreak

\section{Introduction, Motivation, and Objectives}

Example citation: \cite{pivot_mds_R}.
\pagebreak

\section{State of the Art}

\begin{itemize}
    \item R libraries: \verb|Rdimtools|, \verb|dimRed|.
    \item \verb|bigmds| \cite{Delicado2024MDSBigData}.
    \item Landmark Isomap \cite{deSilvaTenenbaum2002}.
    \item Out-of-Core Dimensionality Reduction for Large Data via Out-of-Sample Extensions \cite{reichmann2024outofcoredimensionalityreductionlarge}.
    \item t-SNE Python implementations.
\end{itemize}
\pagebreak

\section{Specification and Design of the Solution}

\begin{itemize}
    \item D\&C algorithm
\end{itemize}
\pagebreak

\section{Development of the Proposal}

\begin{itemize}
    \item DR methods descriptions.
    \item DR methods implementations.
\end{itemize}
\pagebreak

\section{Experimentation and Evaluation of the Proposal}
\pagebreak

\section{Analysis of Sustainability and Ethical Implications}

It must include an analysis of the impact of the following gender-related technical aspects:
\begin{itemize}
    \item issues related to data management and analysis
    \item issues related to equity, where possible biases are identified and assessed both in the data and in the processes carried out in relation to data management and analysis
    \item actions carried out to eliminate or mitigate such biases
\end{itemize}
\pagebreak

\input{main_sections/conclusions.tex}
\pagebreak

\printbibliography
\pagebreak

\section{Annexes}

Example citation: \cite{Rdimtools}.
\pagebreak


\end{document}